\documentclass[a4paper,12pt]{article}

% Packages
\usepackage[utf8]{inputenc}
\usepackage{graphicx}
\usepackage{hyperref}
\usepackage{fancyhdr}
\usepackage{geometry}
\usepackage{titlesec}
\usepackage{listings}
\usepackage{xcolor}

% Page layout
\geometry{a4paper, margin=1in}
\pagestyle{fancy}
\fancyhf{}
\fancyhead[L]{\leftmark}
\fancyhead[R]{\thepage}

% Title formatting
\titleformat{\section}{\Large\bfseries}{\thesection}{1em}{}
\titleformat{\subsection}{\large\bfseries}{\thesubsection}{1em}{}

% Custom code listing style
\definecolor{codebg}{RGB}{245, 245, 245} % Light gray background
\definecolor{keyword}{RGB}{0, 0, 255}    % Blue for keywords
\definecolor{comment}{RGB}{0, 128, 0}    % Green for comments
\definecolor{string}{RGB}{163, 21, 21}   % Red for strings
\definecolor{number}{RGB}{128, 0, 128}   % Purple for numbers

\lstset{
    backgroundcolor=\color{codebg},
    basicstyle=\ttfamily\footnotesize,
    breaklines=true,
    frame=single,
    rulecolor=\color{black},
    tabsize=4,
    showstringspaces=false,
    numbers=left,
    numberstyle=\tiny\color{gray},
    stepnumber=1,
    numbersep=5pt,
    keywordstyle=\color{keyword},
    commentstyle=\color{comment},
    stringstyle=\color{string},
    numberstyle=\color{number},
    morekeywords={x, command1, command2, --option}, % Add custom keywords
    literate=*{0}{{{\color{number}0}}}{1}%
             {1}{{{\color{number}1}}}{1}%
             {2}{{{\color{number}2}}}{1}%
             {3}{{{\color{number}3}}}{1}%
             {4}{{{\color{number}4}}}{1}%
             {5}{{{\color{number}5}}}{1}%
             {6}{{{\color{number}6}}}{1}%
             {7}{{{\color{number}7}}}{1}%
             {8}{{{\color{number}8}}}{1}%
             {9}{{{\color{number}9}}}{1}%
             {:}{{{\color{keyword}:}}}{1}%
             {=}{{{\color{keyword}=}}}{1}%
             {-}{{{\color{keyword}-}}}{1}%
             {>}{{{\color{keyword}>}}}{1},
}

% Document information
\title{User Documentation for X}
\author{Your Name}
\date{\today}

\begin{document}

% Title page
\begin{titlepage}
    \centering
    \vspace*{2cm}
    {\Huge\bfseries User Documentation for X\par}
    \vspace{1.5cm}
    {\Large\itshape Your Name\par}
    \vspace{1.5cm}
    {\large \today\par}
    \vfill
    \includegraphics[width=0.4\textwidth]{logo.png} % Add your software logo here
    \vfill
    {\large Version 1.0\par}
\end{titlepage}

% Table of Contents
\tableofcontents
\newpage

% Introduction
\section{Introduction}
\subsection{Overview}
X is a powerful software designed to [brief description of what your software does]. This document provides a comprehensive guide on how to install, configure, and use X effectively.

\subsection{Features}
\begin{itemize}
    \item Feature 1
    \item Feature 2
    \item Feature 3
\end{itemize}

% Installation
\section{Installation}
\subsection{System Requirements}
Before installing X, ensure that your system meets the following requirements:
\begin{itemize}
    \item Operating System: [e.g., Windows 10, macOS 11, Ubuntu 20.04]
    \item RAM: [e.g., 4GB or more]
    \item Disk Space: [e.g., 500MB of free space]
\end{itemize}

\subsection{Installation Steps}
Follow these steps to install X on your system:
\begin{enumerate}
    \item Download the installer from [website URL].
    \item Run the installer and follow the on-screen instructions.
    \item Once the installation is complete, launch X from the Start Menu or Applications folder.
\end{enumerate}

% Usage
\section{Usage}
\subsection{Getting Started}
To start using X, follow these steps:
\begin{enumerate}
    \item Open X from the Start Menu or Applications folder.
    \item Create a new project or open an existing one.
    \item Use the toolbar and menus to access various features.
\end{enumerate}

\subsection{Basic Commands}
Here are some basic commands you can use in X:
\begin{lstlisting}[language=bash]
# Example command 1
x command1 --option

# Example command 2
x command2 --option
\end{lstlisting}

% Configuration
\section{Configuration}
\subsection{Setting Preferences}
You can configure X by navigating to the Preferences menu:
\begin{enumerate}
    \item Go to \texttt{File > Preferences}.
    \item Adjust the settings according to your needs.
    \item Click \texttt{Save} to apply the changes.
\end{enumerate}

\subsection{Configuration File}
X can also be configured using a configuration file located at \texttt{/path/to/config/file}. Here is an example configuration:
\begin{lstlisting}
[General]
setting1 = value1
setting2 = value2
\end{lstlisting}

% Troubleshooting
\section{Troubleshooting}
\subsection{Common Issues}
\begin{itemize}
    \item \textbf{Issue 1}: Description of the issue and how to resolve it.
    \item \textbf{Issue 2}: Description of the issue and how to resolve it.
\end{itemize}

\subsection{Getting Help}
If you encounter any issues not covered in this document, please contact our support team at \href{mailto:support@x.com}{support@x.com}.

% Appendix
\section{Appendix}
\subsection{Glossary}
\begin{itemize}
    \item \textbf{Term 1}: Definition of term 1.
    \item \textbf{Term 2}: Definition of term 2.
\end{itemize}

\subsection{Additional Resources}
\begin{itemize}
    \item \href{https://www.x.com/documentation}{Official Documentation}
    \item \href{https://www.x.com/forum}{Community Forum}
\end{itemize}

\end{document}
